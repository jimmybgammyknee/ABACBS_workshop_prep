%-------------------------------------------------------------------------------
%  \author Jan P Buchmann <jan.buchmann@sydney.edu.au>
%  \copyright 2018 The University of Sydney
%  \description
%-------------------------------------------------------------------------------

\documentclass{beamer}
\usetheme{default}
\usepackage{inconsolata}
\hypersetup{pdfstartview={Fit}}
\beamertemplatenavigationsymbolsempty
\setbeamertemplate{footline}[frame number]
\usepackage{xspace}
\usepackage{booktabs}
\usepackage{hyperref}
\usepackage{graphics}
\usepackage{listings}
\usepackage{jpb.lst}
\usepackage{multirow}
\usepackage{tikz}
\newcommand{\tty}[1]{\texttt{#1}\xspace}
\newcommand{\compresslist}{
  \setlength{\itemsep}{1pt}
  \setlength{\parskip}{0pt}
  \setlength{\parsep}{0pt}
}

%%% Titleslide  %%%
\title[]{Introdution to Singularity containers}
\subtitle{ABACBS 2018 Container Workshop\\James Breen \& Jan P. Buchmann}
\author[]{Jan P. Buchmann\\\small{\href{mailto:jan.buchmann@sydney.edu.au}{jan.buchmann@sydney.edu.au}}}
\institute{The University Of Sydney}
\date{2018-11-29}
%**************************
\begin{document}
  \titlepage

  \begin{frame}{Just in case...}
    \footnotesize
    \begin{description}
      \item[Download] \leavevmode\\ \url{https://www.sylabs.io/singularity/get-singularity/}
      \item[Manual] \leavevmode\\ \url{https://www.sylabs.io/guides/3.0/user-guide.pdf}
      \item[Paper] \leavevmode\\ \url{https://doi.org/10.1371/journal.pone.0177459}
    \end{description}
  \end{frame}

  \begin{frame}{Singularity: Containers for HPC}
   Containers are encapsulated system enviroments
    \begin{description}
      \item[Not a microservice:] Scientific focus, e.g. whole pipelines
      \item[Single file:] The image is a single file easily share, archive,
                          reproduce, good for parallel file sytems, e.g. Lustre
      \item[Run as user:] root to create, user to run
      \item[Access HPC resources:] MPI, GPUs, InfiniBand/Network, file systems
    \end{description}
  \end{frame}

  \begin{frame}{Biggest difference to Docker}
    \small
    \begin{block}{Privileges}
      You run the container as the user who invokes singularity. You can only
      be root in the container if you run it as root. Not your usual HPC
      experience.
    \end{block}
    \begin{block}{No daemon}
      There is no daemon required, Singularity image is mounted as a loopback.
      Docker swarms need a DockerEngine on each  node or instance they run.
    \end{block}
    \begin{block}{Runs closer to the host}
      Running a singularity container bind mounts your \tty{\$HOME}, \tty{/dev},
      \tty{/sys}, \tty{/tmp}, \tty{/var/tmp}, and \tty{/proc} by default.
    \end{block}
    \begin{block}{Expect ongoing chnages}
      Expect ongoing changes, e.g. differences singularity 2.2.5 to 3.0.1.
      Keeps you on your toes.
    \end{block}
    \begin{block}{Docker layers change}
      Docker image layers can change, i.e. a docker image has likely changed
      layers when pulled a couple of months later.
    \end{block}
  \end{frame}

  \begin{frame}{Speed}
    Negligible tradeoffs compared to bare metal.
    \begin{itemize}
      \item \url{https://arxiv.org/pdf/1709.10140.pdf}
    \end{itemize}

  \end{frame}

  \begin{frame}{Overall Singularity workflow}
    \begin{enumerate}
      \item Build basic image
      \item Modify basic image $\rightarrow$ your image
      \item Copy your image to HPC, laptop, Grandma's fridge...
      \item Run your image
    \end{enumerate}
  \end{frame}

  \begin{frame}{Building singularity containers}
    \footnotesize
    \begin{block}{Docker Hub (docker://)}
      \tty{singularity build lolcow.simg docker://godlovedc/lolcow}
    \end{block}

    \begin{block}{Container Library  (library://)}
      requires $\ge 3.0$\\
      \tty{singularity build lolcow.simg library://sylabs-jms/testing/lolcow}
    \end{block}
    \begin{block}{Singularity Hub (shub://)}
        {singularity build demo.simg shub://jasongallant/singularity\_demosingularity}
    \end{block}
    \begin{block}{Singularity receipe files}
      Roll your own
    \end{block}
  \end{frame}

  \begin{frame}{Invoking singularity}
    \begin{block}{Run singularity}
      \footnotesize
      \texttt{singularity [global options] command [command options]}\\
      For example:\\
      \texttt{singularity -v build --sandbox /tmp/ubuntu docker://ubuntu:latest}
    \end{block}
    \begin{block}{Very useful commands}
      \texttt{\$: singularity help}\\
      For example:\\
      \texttt{\$: singularity help build}\\
    \end{block}
  \end{frame}

  \begin{frame}{Workflow: Build environment (root)}
    \small
    \begin{block}{Interactive}
      \bashexternal{lsts/singularity-interactive.sh}
    \end{block}
  \end{frame}

  \begin{frame}[fragile]{Workflow: Let's build ourself a container}
  \begin{block}{In your terminal}
  \begin{verbatim}
$: cd repo/singularity
$: mkdir work
$: cd work
\end{verbatim}
  \end{block}
    Steps: \url{../commands/commands.md}
  \end{frame}

  \begin{frame}[fragile]{Workflow: Recipe file}
      \begin{itemize}
        \item Preferred way to crate images. Use interactive to get details
              right, test depencdencies, etc.  than fixeverything in a receipe
              file.
        \end{itemize}
      Example: \url{../recipes/raxml.ubuntu.def}
    \begin{block}{Run}
    \footnotesize{}
    recipes
      \begin{verbatim}
$: cd repo/singularity/work
$: sudo singularity build ..recipes/raxml.ubuntu.def raxml.ubuntu.simg
      \end{verbatim}
    \end{block}
  \end{frame}

  \begin{frame}{Singularity on HPC}
    \begin{itemize}
      \item Host and container OS need same OpenMPI/MPI version
      \item Adjust image for binding directories
      \item Adjust GPU specific files, libraries
      \item Very useful: \url{https://hpc.nih.gov/apps/singularity.html}
    \end{itemize}
    Example PBS file: \url{../recipes/raxml-sing.pbs}
  \end{frame}

  \begin{frame}{Possible error}
    \begin{block}{FATAL: kernel too old}
      The glibc version in the image is too new for the host kernel.
      Update host kernel (not always possible) or use image with
      older glibc.
    \end{block}
    \begin{block}{FATAL: Unable to create build: unable to get conveyorpacker: invalid build source}
      Can occur with newer version (4537182). Bootstrap needs to br first line.
      Remove all comments until Bootstrap:...\\
      (\tty{../examples/\{raxml.ubuntu.def,raxml.ubuntu.git.def\}})
    \end{block}
  \end{frame}
\end{document}
