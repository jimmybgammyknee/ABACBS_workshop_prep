%-------------------------------------------------------------------------------
%  \author Jan P Buchmann <jan.buchmann@sydney.edu.au>
%  \copyright 2018 The University of Sydney
%  \description
%-------------------------------------------------------------------------------

\documentclass{beamer}
\usetheme{default}
\usepackage{inconsolata}
\hypersetup{pdfstartview={Fit}}
\beamertemplatenavigationsymbolsempty
\setbeamertemplate{footline}[frame number]
\usepackage{xspace}
\usepackage{booktabs}
\usepackage{hyperref}
\usepackage{graphics}
\usepackage{listings}
\usepackage{multirow}
\usepackage{tikz}

\newcommand{\compresslist}{
  \setlength{\itemsep}{1pt}
  \setlength{\parskip}{0pt}
  \setlength{\parsep}{0pt}
}

%%% Titleslide  %%%
\title[]{Introdution to Singularity containers}
\subtitle{}
\author[]{Jan P. Buchmann\\\small{\href{mailto:jan.buchmann@sydney.edu.au}{jan.buchmann@sydney.edu.au}}}
\institute{The University Of Sydney}
\date{2018-11-29}
%**************************
\begin{document}
  \titlepage

  \begin{frame}{Just in case...}
    \begin{description}
      \item[Download] \url{https://www.sylabs.io/singularity/get-singularity/}
      \item[Manual] \url{https://www.sylabs.io/guides/3.0/user-guide.pdf}
    \end{description}
  \end{frame}

  \begin{frame}{Singularity: Containers for HPC}
   Containers are encapsulated system enviroments
    \begin{description}
      \item[Not a microservice:]Scientific focus, e.g. whole pipelines
      \item[Single file:] The image is a single file easily share, archive,
                          reproduce, good for parallel file sytems, e.g. Lustre
      \item[Run as user:] root to create, user to run
      \item[Access HPC resources:] MPI, GPUs, InfiniBand/Network, file systems
    \end{description}
  \end{frame}


  \begin{frame}{Biggest difference to Docker}
    \begin{block}{Privileges}
      You run the container as the user who invokes singularity. You can only
      be root in the container if you run it as root. Not your usual HPC
      experience.
    \end{block}
    \begin{block}{No daemon}
      There is no deamon required, Singularity image is mounted as a loopback.
      Docker swarms need a DockerEngine on each  node or instance they run.
    \end{block}
    \begin{block}{Runs closer to the host}
      Running a singularity container bind mount your \$HOME, /dev, /sys, and
      /proc automatically by default.
    \end{block}
    \begin{block}{Singularity Image Format (SIF) ($\geq$3.0)}
      Image container format resembling a general file system whoch will allow
      PGP signing, block encryption, partitions accomodating multiple OSes,
      fast metadata access.
    \end{block}
    \begin{block}{Docker layers change}
      Docker image layers can change, i.e. a docker image has likely changed
      layers when pulled a couple of months later.
    \end{block}
  \end{frame}

  \begin{frame}{Speed}

  \end{frame}

  \begin{frame}{Overall singularity workflow}
    %  -------------------------------------------------------------------------------
%  \author Jan P Buchmann <jan.buchmann@sydney.edu.au>
%  \copyright 2018 The University of Sydney
%  \description
%  -------------------------------------------------------------------------------
\usetikzlibrary{shapes.geometric, positioning, calc, shadings}
\begin{tikzpicture}[auto,font=\small,
                    every node/.style={node distance=1cm},
                    command/.style={rectangle, font=\scriptsize\ttfamily}]
    \def\blockdist{0.3}
    \node (container) [command]                        {singularity build ... image.sif};
    \node (laptop)    [below = of container]        {Laptop};
    \node (hpc)       [left = of laptop]        {HPC};
    \node (iot)       [right = of laptop]        {Granny's fridge};
     \node (image)    [command, below = of laptop] {singularity run/exec/shell image.sif};

    \path[->] (container) edge [] (laptop)
                          edge [] (hpc)
                          edge [] (iot);
    \path[->] (laptop)    edge [] (image)
              (hpc)       edge [] (image)
              (iot)       edge [] (image);
\end{tikzpicture}

% The comment style is used to describe the characteristics of each force
%comment/.style={rectangle, inner sep= 5pt, text width=4cm, node distance=0.25cm, font=\scriptsize\sffamily},
% The force style is used to draw the forces' name
%force/.style={rectangle, draw, fill=black!10, inner sep=5pt, text width=4cm, text badly centered, minimum height=1.2cm, font=\bfseries\footnotesize\sffamily}]

  \end{frame}

  \begin{frame}{Building singularity containers}
    \begin{description}
      \item[Docker Hub (docker://)] singularity build lolcow.simg docker://godlovedc/lolcow
      \item[Container Library (library://)] singularity build lolcow.simg library://sylabs-jms/testing/lolcow
      \item[Singularity Hub (shub://)] singularity build demo.simg shub://jasongallant/singularity\_demosingularity
      \item[Singularity receipe files] Roll your own
    \end{description}
  \end{frame}

  \begin{frame}{Invoking singularity}
    \footnotesize
    \texttt{singularity [global options] command [command options]}\\
    \texttt{singularity -v build --sandbox /tmp/ubuntu docker://ubuntu:latest}
  \end{frame}

  \begin{frame}{Workflow: Build enviroment (root)}
        \begin{block}{Interactive}
          \begin{itemize}
            \item singularity build --sandbox /tmp/ubuntu docker://ubuntu:latest
            \item singularity exec --writeable /tmp/ubuntu apt-get install python
            \item singularity build /tmp/ubuntu.sif /tmp/ubuntu
          \end{itemize}
        \end{block}
  \end{frame}
  \begin{frame}{Workflow: Build enviroment (root)}
    \begin{block}{Receipe}

    \end{block}
  \end{frame}
  \begin{frame}
    \begin{block}{Production enviroment (user)}

    \end{block}
  \end{frame}

  \begin{frame}{singularity receipe files}
  \end{frame}

  \begin{frame}{singularity on HPC}
    \begin{itemize}
      \item host and container need same openmpi version
      \item adjust image for binding
    \end{itemize}
  \end{frame}

  \begin{frame}{Bind paths}

  \end{frame}

  \begin{frame}{Overlays}
  \end{frame}


\end{document}
